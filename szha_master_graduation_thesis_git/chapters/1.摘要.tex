\begin{abstract}{实体链接,主动学习,语料构建}
	实体链接是将文本中包含的命名实体指称链接到知识库对应实体条目的任务。经过近十年的发展,监督学习方法和无监督学习方法都在实体链接任务中得到了广泛的应用。然而,关于如何快速训练实体链接模型和构建高质量实体链接语料库的工作相对较少。主动学习能够根据学习进程,选择最佳样本交由人工标注,在减少训练样本数量的同时,保持或提高模型性能。因此本文主要研究基于主动学习的实体链接方法。
	
	本文分析了实体链接任务的特点,基于主动学习,对实体链接任务中模型训练和语料构建的方法做了相关研究,主要工作包括:
	
	(1)针对基于监督学习的实体链接模型,本文通过主动学习减少人工标注样本的数量,并提出基于流行度的初始样本选择方法以及基于综合不确定度和流行度的迭代训练样本选择方法。在初始训练样本选择阶段,保证了初始训练样本的代表性。在后续迭代训练阶段,兼顾了被选择样本的不确定度和代表性。
	
	(2)针对基于无监督学习的实体链接银标准语料构建任务,为提高标注质量,本文提出了基于主动学习和无监督学习的样本标注方法。该方法能够选择未标注样本中信息量较大的若干样本交由人工标注,已标注指称项通过证据传播提高语料整体的标注正确率。
	
	实验表明,使用基于流行度的初始样本选择方法以及基于综合不确定度和流行度的迭代训练样本选择方法对加快实体链接模型训练的收敛速度有显著的效果;使用基于主动学习和无监督学习的样本标注方法在标注较少指称项的前提下可以有效提升银标准语料的标注正确率。
\end{abstract}
\begin{englishabstract}{Entity Linking, Active Learning, Corpus Construction}
	Entity linking is the task of determining the identity of entities mentioned in text. Supervised learning approaches and unsupervised learning approaches have been widely used in entity linking task in the past decade. However, only a few studies have been reported on accelerating model training and improving corpus construction. Active learning can contribute to interactively obtain optimum samples and provide them for annotators to conduct manual annotation according to the learning process. Meanwhile, it can reduce the quantity of the training samples as well as keep or improve model performance.
	
	This thesis analyzes the characteristics of entity linking task, and use active learning approaches to handle model training and corpus construction task based on active learning.
	
	The main contributions of this thesis include two aspects as follow:
	
	(1) In consideration of supervised learning model of entity linking, this thesis reduces human annotating effort by using active learning, and proposes two approaches. One is an initial sample selection approach based on popularity, as known as sampling by popularity (SBP) . The other is an iterative training sample selection approach based on comprehensive uncertainty and popularity, as known as sampling by uncertainty and popularity (SUP) . This way ensures representative of initial training sample in the initial sample selection stage and considers both uncertainty and representative of selected samples in the following stage of iterative sample training.
	
	(2) To construct entity linking corpus, this thesis proposes an annotating approach based on active learning and unsupervised learning for improving annotation quality. In this way, the most informative samples of unlabeled mentions can be found for annotators to annotate while the precision rate of the whole corpus can be improved by propagating the evidence of labeled mentions.
	
	Experiments in this thesis show two main points. One is that approaches of SBP and SUP can effectively accelerate the training process of entity linking model. The other is that approaches of annotation based on active learning and unsupervised learning can effectively improve the accuracy of annotating a silver-standard entity linking corpus on the premise of annotating fewer mentions.
\end{englishabstract}