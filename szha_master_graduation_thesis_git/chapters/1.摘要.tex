\begin{abstract}{地理高考,本体构建,知识库问答,双向长短期记忆网络,注意力机制}
	在人工智能领域,自动解答高考题是一项很具挑战的任务。与一般事实性问答的问题不同,高考题带有很强的选拔性。其问题考察形式多变,其答案求解往往不能一步得到,通常需要做进一步的知识推理。在辅助解答高考地理题时,目前面临两个问题:第一是缺乏高度结构化的地理核心知识库,第二是地理问题表达形式多样,导致问题理解困难。针对以上两个问题,本文作了如下工作:
	
	(1)为解决高度结构化的地理核心知识库缺乏问题,本文构建了中文地理本体(Chinese  Geographic Ontology, CGeoOnt)。该本体以人教版高中地理教科书为知识源,使用万维网本体语言(Web Ontology Language, OWL)为知识表示语言,以课本章节为知识体系,人工总结其核心地理概念、地理关系、地理考点,并将其表示为本体形式。同时,本文将构建的CGeoOnt与本体Clinga进行本体融合,得到一个更大规模的中文地理本体知识库。
	
	(2)为解决地理问题问法多样导致其难以理解问题,本文使用基于注意力机制的知识库问答模型。该模型以双向长短期记忆网络为基础问答模型,结合注意力机制对地理问题、答案进行表示,答案中每个词的向量生成,均结合其对问题各词的注意力权重分配,使答案可以更好的对齐问题中关键信息,减弱无效信息的干扰,因此更易区分正确答案和错误答案。实验表明,该问答模型对于辅助解答地理高考题具有很好的应用价值。
	
	(3)为解决中文地理问答模型在训练和测试中数据集缺失问题,本文从互联网收集了一个问法多样的中文地理问题集。本文使用百度问题推荐以及百度搜索API,以本体知识库高频核心知识三元组为数据源,依次访问到二十万个Web地理问题,然后半自动加人工挑选出其中的有效问题,再人工从知识库寻找问题答案,形成最终地理问答数据集。
\end{abstract}
\begin{englishabstract}{Geography College Entrance Examination, Ontology Construction, Knowledge Base Question Answering, Bi-LSTM, Attention Mechanism}

	In the field of artificial intelligence, it is a challenging task to automatically answer the college entrance examination (namely Gao Kao) questions. Different from the questions of the general factoid question answering, the questions of GaoKao have a strong selection, also the problem's forms are changeable, and its solution often cannot be obtained in one step, and it usually needs further knowledge inference. At present, there are two problems in assisting in solving the geography problem of GaoKao: The first is the lack of a highly structured geographical core knowledge base, and the second is that geographical problems are expressed in various forms, which make it hard to understand.In view of the above two questions, this article has done the following works:
	
(1) In order to solve the problem of the lack of the highly structured geo-core Knowledge Base(KB), this paper constructs the Chinese Geographical Ontology(CGeoont). The ontology is based on the human-educational version of the geography textbook for the knowledge source, uses the World Wide Web Ontology language(OWL) as the knowledge representation language, takes the textbook chapter as the Knowledge system structure, summarizes its core geographical concept, the geographical relation, the geographical Test center, and expresses it as the ontology form. At the same time, this paper integrates the constructed CGeoont with ontology Clinga, and obtains a more large-scale Chinese geographical ontology knowledge Base.
	
(2) In order to solve  the problem of understanding various and changeable formed geo-question, we employ the Knowledge base question and answering model based on attention mechanism. The model is based on bidirectional long and short term memory network, according to the attention mechanism, the geographical problem and the answer are expressed, the vector generation of each word in the answer is combined with its attention weight distribution, so that the answer can better align the key information in the problem, weaken the interference of invalid information, It makes it easier to distinguish between correct answers and wrong answers. The experimental results show that this model is very useful for solving the problem of Geography entrance examination.

(3) To solve the problem of the lack of data set in the training and testing of Chinese geography question and answer model, this paper collects a variety of Chinese geography problem sets from the Internet. This article uses the Baidu question recommendation API as well as the Baidu Search API, takes the ontology Knowledge Base high frequency core knowledge triples as the data source, then accesses to 200,000 web geography question, then picks out the effective questions semi-automaticly and manually, then manuallyseeks the question answer according to the knowledge base , forms the final geography question and answering dataset.

\end{englishabstract}