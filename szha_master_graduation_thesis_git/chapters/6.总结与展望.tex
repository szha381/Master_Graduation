\chapter{总结与展望}
\section{总结}
本文研究基于主动学习的实体链接方法,解决当前存在的两个问题,并通过实验进行验证。

(1)使用监督学习模型处理实体链接任务时,需要人工标注足够的训练样本集,此项工作费时费力。

基于主动学习的实体链接监督模型训练方法可有效解决此问题。使用基于流行度的初始样本集选择方法提高初始分类器的性能,使用综合不确定度和流行度的样本选择方法加快迭代训练过程中模型的收敛速度。

在基于主动学习的实体链接监督模型训练方法的实验环节中,给定初始样本集较小时,本文提出的初始样本选择方法的初始分类器性能比随机选择的基线方法的性能提升了 10.1\%。在后续迭代训练过程中,本文提出的综合不确定度和流行度的样本选择方法相比传统的基于不确定度的选择算法的$deficiency$值降低了 16.1\%。

(2)在成本有限的情况下,传统的标注方法对于构建高质量的实体链接银标准语料,效率低下。

可采用基于主动学习的实体链接银标准语料构建来解决此问题。借助基于图的协同推断方法对未标注样本中的指称项进行预标注,然后通过主动学习方法选择预标注不确定度高的指称项进行人工标注,以此提高错误预标注的命中率,提高人工标注效率。另一方面,通过标注证据传播,将人工标注结果传播到未标注指称项,从而减少人工标注的工作量。

在基于主动学习的实体链接银标准语料构建方法的实验环节中,本文对主动学习的待标注样本选择方法进行了改进,并融入了标注证据传播,加速了主动学习进程。在仅标注50\%的指称项的情况下,实验中性能最佳的方法相比于非主动学习的顺序标注方式,标注语料正确率提升了约8个百分点。

\section{未来展望}
从实验结果看,本文提出了一系列基于主动学习的实体链接方法,但仍存在值得改进的地方。

(1)在基于有监督学习的实体链接模型训练过程中,考虑到使用的特征比较简单,后续工作可以考虑用分布式词表示方法计算指称项与候选实体之间的相似度。

(2)实体链接模型采用的是单个模型的训练方式,模型性能在一定程度上受到模型选择的制约。未来工作中,我们将会使用多个模型来处理实体链接任务,并基于委员会的主动学习方法加速模型的训练,利用各种模型的差异来解决单个模型具有制约性的缺陷。

(3)在基于主动学习的实体链接银标准语料构建方法中,本文主要考虑的是文档内指称项之间的关系,文档与文档之间是相互独立的。而实际上,不同文档之间的指称项也可能存在相互关联的关系。如何处理这种跨文档的指称项关系检测,对提升语料标注效率具有重要意义,这也是未来工作之一。