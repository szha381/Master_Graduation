\begin{abstract}{\TeX, \LaTeX, 文档类, 学位论文}
	实体链接是将文本中包含的命名实体指称链接到知识库对应实体条目的任务。经过近十年的发展,监督学习方法和无监督学习方法都在实体链接任务中得到了广泛的应用。然而,关于如何快速构建高质量实体链接语料库的工作则相对较少。主动学习能够根据学习进程,选择最佳样本交由人工标注,因此本文基于主动学习研究实体链接方法。
	
	本文分析了实体链接任务的特点,基于主动学习的方法,对实体链接中训练样本选择的方法做了相关研究,主要工作包括:
	
	(1)针对基于监督学习的实体链接模型,为减少人工标注样本的数量,本文提出了基于流行度的初始样本选择方法以及综合不确定度和流行度的迭代训练样本选择方法。在初始训练样本选择阶段,保证了初始训练样本的代表性。在后续迭代训练阶段,兼顾了被选择样本的不确定度和代表性。
	
	(2)针对基于无监督学习的实体链接模型,为提高标注质量,本文提出了基于主动学习和半监督学习的样本标注方法。该方法能够选择未标注样本中信息量较大的若干样本交由人工标注,已标注样本的命名实体指称和目标实体通过证据传播提高语料整体的标注正确率。
	
	实验表明,使用基于流行度的初始样本选择方法以及综合不确定度和流行度的迭代训练样本选择方法对加快实体链接模型训练的收敛速度有显著的效果;使用基于主动学习和半监督学习的样本标注方法在命名实体标注中取得了较好的结果。
\end{abstract}
\begin{englishabstract}{\TeX, \LaTeX, document class, thesis/dissertation}
	This work presents an introduction of how to use \seuthesix document class to 
	typeset the thesis/dissertation of Southeast University.
\end{englishabstract}