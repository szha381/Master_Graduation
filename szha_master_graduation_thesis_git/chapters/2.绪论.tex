\chapter{绪论}
\section{研究背景}
随着互联网技术的飞速发展,网络已逐渐成为信息的主要载体之一。这些信息主要以自然语言文本的形式存在,如新闻、博客、评论、搜索引擎的查询关键词等。这些文本包含着大量的命名实体(Named Entity),如人物(Person)、地点(Location)、机构(Organization)等。由于自然语言具有多义性,同一个实体指称项在不同的上下文环境中可能指向不同的实体。例如,以下两个摘自维基百科词条的句子都包含“Michael Jordan”这个实体指称项:

(1)Michael Jordan, the American retired professional basketball player,  starred in the 1996 feature film Space Jam.

(2)Michael Jordan is an American scientist, Professor at the University of California, Berkeley and leading researcher in machine learning, statistics, and artificial intelligence.

\noindent 在上面这两个句子中,同一个实体指称项“Michael Jordan”在各自不同的上下文环境中分别指向\textit{Michael Jeffrey Jordan}(前NBA公牛队球员)和\textit{Michael I. Jordan}(知名计算机科学和统计学学者)。

实体链接(Entity Linking)任务将文本中包含的指称项关联到知识库(如Wikipedia,DBpedia)中对应的条目。实体链接作为自然语言处理(Natural Language Processing)的基础工作之一,对信息抽取(Information Extraction)\cite{IEFMPDD}、知识库填充 (Knowledge Base Population) \cite{MarkDredze}、 问 答 (Question Answering)\cite{Yih}等后续任务来说具有重要意义。对于上述两个句子,可根据实体指称项上下文与候选实体摘要文本的语义相似度等特征,利用机器学习模型从候选实体集中找出最佳目标实体。

在实体链接任务中,监督学习(Supervised Learning)得到了广泛的应用\cite{Xiaohua,LLEWKB,ELNDSCM}。考虑到网络资源的快速增长,未标注数据的获取是相对容易的,但是对未标注数据进行人工标注却非常费时费力。并且文本所属领域、知识库的选择以及网 络中不断涌现出来的新实体等对系统性能都会有一 定的影响。因此,降低训练样本标注工作量对于开 发和维护实体链接系统具有重要意义。然而目前对如何降低实体链接训练样本标注工作量的研究却很少。另外,研究者对无监督学习(Unsupervised Learning)在实体链接任务中的应用也做了大量的研究\cite{ELKTULASR,UAKGQCL}。与有监督学习相比,无监督学习的好处在于不需要对实体链接样本集做标注,节省了标注工作量。所以,无监督学习模型可用于实体链接银标准(Silver Standard)语料的辅助标注任务。然而,基于无监督学习的实体链接语料辅助标注存在的问题是,标注质量取决于无监督模学习模型性能的好坏,若无监督学习模型性能较差,则标注质量也差。

为了解决上述问题,本文在实体链接任务中引入主动学习(Active Learning)方法。在监督学习任务中,主动学习器能够主动选择信息量大的未标注样本交由人工标注,从而在较小的带标注样本集中训练得到较高性能的模型。基于主动学习的监督学习方法已经应用于多种自然语言处理任务,如命名实体识别(Named Entity Recognition)\cite{ALMNERICT}、文本分类(Text Classification)\cite{SCALRHRTC}、词义消歧(Word Sense Disambiguition)\cite{ALSWSDIM}等。在无监督学习任务中,主动学习器能够选出分类不确定度大的样本,并将其交由人工标注,根据标注结果对所有未标注样本重新分类,从而在尽可能标注较少样本的前提下提高整体样本的分类正确率\cite{UALHGC}。然而,不同于其他自然语言处理任务,实体链接任务有其特点,直接将传统主动学习应用于实体链接任务,效果未必达到最佳。

由于目前主动学习在实体链接任务中的研究很少,本文尝试在基于监督学习的实体链接方法中,使用主动学习选择待标注样本,减小标注样本集的规模,并根据实体链接任务的特点,对已有主动学习的样本选择方法做了改进。在无监督学习的实体链接方法中,本文尝试通过主动学习方法选择分类最不确定的样本,通过标注这些样本,在尽可能在成本有限的条件下提高整体语料的标注质量。

\section{研究内容}
本文的主要研究内容是基于主动学习的实体链接方法,包括基于监督学习的实体链接的待标注样本选择方法、基于无监督学习的实体链接银标准语料构建方法。本文针对实体链接任务中指称项分布不均衡以及相互关联的特点,对已有的主动学习方法进行了改进,在监督学习的实体链接任务中提出了考虑样本代表性的新的样本选择方法;在无监督学习的实体链接任务中提出了考虑指称之间关系的证据传播(Evidence Propagation)方法。

本文的主要研究内容如下:

(1)在有监督学习方法基础上,针对实体链接任务,本文在初始训练样本选择阶段提出了基于指称项流行度的样本选择方法(Sample by Popularity, SBP);在迭代训练样本选择阶段,提出了综合不确定度和流行度的样本选择方法(Sample by Uncertainty and Popularity, SUP)。并与已有主动学习样本选择方法进行了比较,给出了实验结果和分析。

(2)在无监督学习方法的基础上,针对实体链接任务,本文使用主动学习方法选择需要人工标注的样本。并提出了基于指称项字符串相似度(Propagation By Name)、基于指称项语义相似度的证据传播方法(Propagation By Semantic),并给出了实验结果。

\section{论文组织}
本文共分为五章,各章的主要内容如下:

(一)第一章主要介绍相关的基本概念、应用背景,以及研究内容和论文的组织结构。

(二)第二章主要介绍实体链接任务和主动学习方法的研究现状。介绍了实体链接任务的基本概念以及常用的处理方法,分析了目前实体链接任务方法各自的优缺点;还介绍了主动学习方法的概念、常用的方法、以及它在其它机器学习任务中的应用。

(三)第三章首先介绍基于主动学习的实体链接模型的训练方法。然后介绍经过改进的主动学习方法,提出了基于流行度的初始样本选择方法以及综合不确定度和流行度的迭代训练样本选择方法。并通过实验验证了本文所提方法的有效性。

(四)第四章首先介绍基于主动学习的实体链接银标准语料构建方法。然后提出待标注样本的选择方法,以及标注结果的证据传播方法。并通过实验验证银标准语料库构建效率的提升效果。

(五)第五章对全文进行总结,指出本文的创新点和不足之处,并对未来的研究进行展望。