\chapter{总结与展望}
\section{总结}
本文为辅助解答地理高考题所做的工作,主要解决了在辅助解答地理高考题过程中的两个问题,第一个是缺乏高质量的地理核心知识库,第二个是地理问题表达多样而导致的问理解困难问题。

为解决缺乏高质量的地理核心知识库的问题,本文以高中地理教科书和北京市十年高考地理试题为知识源,通过自底向顶的本体构建思想,先从十年高考题中提炼出地理核心考点,然后去地理教材中寻找可以解答核心考点的核心地理知识,以地理教科书中的知识组织框架为大体的本体知识层次组织,并通过描述能力很强的本体语言OWL表示地理核心知识,得到高质量、高度结构化的地理本体CGeoOnt。同时,本文还将863项目组以百度百科自动构建的中文地理本体Clinga与CGeoOnt融合得到目前中文地理领域规模最大、质量最高的地理本体知识库,用于辅助地理高考解题。

为解决地理问题表达多样而导致的问理解困难问题,本文以 Clinga、CGeoOnt 为中文地理核心知识库,构建了一个问答系统。该问答系统以 Attention-based Bi-LSTM 为问答模型,分别用两个共享参数的 Bi-LSTM 网络来表示问题和答案。同时,答案的表示结合答案序列对问题的注意力权重,模型使用有标记的问答对做训练,在从 web收集到的多样性的地理问题数据集上,问答指标 MRR、Accuracy@N 分别达到 0.834、0.872的好结果, 当一个问题以不同方式提问时,本文模型同样具有很好的适应性。

本文构建的问答系统致力于辅助解答高考地理题,对于单实体单关系的地理问题,即使该问题提问方式形式多样,实验表明本系统仍可以比较好的解决,因此对于解答高考地理题有一定的实用价值。

\section{未来展望}
本文构建地理本体CGeoOnt采用的是人工构建的方式,从整体来看虽然构建的本体质量比较高,但是总体时间花费过大,六人标注团队一年半的时间仅仅标注了两万多条,因此,可以尝试使用自动或者半自动加人工的方式来构建地理本体,并且开发合适的地理知识标注工具,提高地理标注效率。

再者,对于本文构建的地理知识库问答系统,从模型上看,对于答案的表示,除了本文结合问题中词级别的注意力机制,模型还可以结合地理题本身的特征,如地理问题答案类型、答案关系等,或者还可以结合知识库本体本身丰富的语义特征,将实体的上下文类别语义信息添加到问题、 答案的表示中,从这些方面更加完善地对问题、 答案进行表示,从而提升问答的性能。

